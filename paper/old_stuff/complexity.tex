\section{Complexity}
\label{section:complexity}
\textit{Q5. Complexity: What is the complexity of an artifact under review and what is the simplest way of measuring this complexity?}\\

Contributions are sent to the mailing list in the form of diffs for review.
Measuring the complexity of a change differs from measuring the complexity of a
file or the entire system \cite{German2005SM}. For example, changing a large
number of simple functions across multiple files may be as difficult as
changing the code contained within a single complex function.
%
Traditional complexity measures often require the whole file or system and were
not designed to measure the complexity of code fragments \cite{Hindle2008ICPC}.
Further, multiple studies have found that these measures correlate very
strongly with the lines of code in a file
\cite{Graves2000IEEE,Leszak2002JSS,Herraiz2007MSR} and may not provide much
additional information about the complexity of the system.

In this section, we provide various possible measures of change complexity.
Our measures are applied to patch contributions~\cite{German2005SM}.
%
All of our measures are based upon the premise that changes that are farther
from each other or involve multiple entities are more complex than those closer
together involving fewer entities. 
%
In total, we measure change complexity in seven ways:

\begin{itemize}

\item  the churn or change size (see previous section \ref{section:size}), 

\item the number of modified files in a diff, 

\item the number of diffs per review, 

\item the number of and distance between contiguous change blocks within a diff
(\ie hunks), 

\item the directory distance between these files, and 

\item depth of indentation of a change. 

\end{itemize}

%
Since the goal of each measure is the same, to assess the complexity of a
change, we expect some of our measures to be highly correlated. 
%we want the simpliest most parsimonious measures. 
In the interest of parsimony, we will select the simplest set of measures that
adequately captures change complexity. Future work could examine each of these
measure outside of the context of peer review. Given the goals of this work,
when appropriate, instead of providing a box plot, we show that a more complex
measure correlates with a simpler measure\footnote{We use non-parametric
Spearman correlations, correlation are statistically significant at $ p < .001$
unless otherwise specified}. In this case, the complex measure is dropped from
further analysis. 

{\bf Files and diffs:} we count the number of files contained in a review and
the number of diffs. The posting of a diff often represents a re-write of the
original contribution or a sequence of diffs that implement a larger feature.
As more files change and more diffs are produced, reviewers will likely need an
understanding of a larger proportion of the system.
%-- see discussion below and Figures \ref{fig:rtc_files} and
%\ref{fig:ctr_files}. 

Figures \ref{fig:rtc_files} and \ref{fig:ctr_files} show the number of files
modified per review. For {\bf RTC}, there are one to two files modified in the
median case and between three and five at the $75^{th}$ percentile.  For {\bf
CTR}, there is one and between three and four for the median and $75^{th}$
percentile, respectively. There is a moderate correlation between churn and the
number of files changed (min (kde ctr) $= .56$, mean $= .64$, max (lx)
$=.74$). Since Linux is the only project that shows a strong correlation with
churn, we keep the number of changed files as a predictor, but also add an
interaction term between churn and number of files (See models in Section
\ref{section:model}).

\begin{figure}
\centering
\includegraphics[width=\plotImageWidth]{figures/rtc/files} 
\caption{RTC -- Number of modified files per contribution}
\label{fig:rtc_files}
\end{figure}
\begin{figure}
\centering
\includegraphics[width=\plotImageWidth]{figures/ctr/files}
\caption{CTR -- Number of modified files per contribution}
\label{fig:ctr_files}
\end{figure}

The number of diffs per review is low. For {\bf CTR} at the $90^{th}$
percentile it is one, indicating that few subsequent diffs are posted during
CTR. For {\bf RTC} it is one at the median and between one and two at the
$75^{th}$ percentile, indicating that posting updated diffs is also rare with
RTC. The correlation between the number of diffs and the number of files is
weak (min (kde ctr) $=.11$, mean $= .26$, max (lx) $= .57$) as is the
correlation with churn (min (kde ctr) $= .10$, mean $= .25$, max (lx) $= .54$).
The number of diffs is kept as a predictor.

{\bf Change blocks:} we measure the number of contiguous change blocks, or
hunks, and the distance between these blocks in a modified file and sum them
across the entire review.  Contiguous changes  are likely easier to review than
changes that are far apart and that are potentially in multiple functions
within a file.  
%
We find that the number of change blocks correlates strongly with a simpler
measure, churn (min (kde) $= .70$, mean $= .81$, max (gn) $= .86$).
%Just c,cpp (mean $= .79$, kde $= .69$ to svn $= .84$).
Intuitively and in practice, it appears that as the number of changed blocks of
code increases, so to does the number of total changes, or churn. The distance
between change blocks is dropped from further analysis, but could be
investigated in future work with different objectives.

%
{\bf Directory distance:} we measure the directory distance between the files
in a review. The premise for this measure is that files that perform similar
functions are typically closer in the directory hierarchy than files that
perform dissimilar functions~\cite{Bowman1999ICSE}, thus the directory
structure loosely mirrors the system architecture~\cite{RigbySubTOSEM}. The
distance of two files in the same directory have a distance of zero, while the
distance for files in different directories is the number of directories
between the two files in the directory hierarchy.  We expect that changes that
involve files that are far apart will crosscut the functionality of a system
and be more complex to review.
%
We find that the directory distance of the files in a change correlates
strongly with a simpler measure, the total number of modified files in a review
(min (fb rtc $= .78$, mean $= .87$ max (fb ctr) $= .92$).  Intuitively we can
see that as the number of files increases, the chance that two files will be
further apart in the directory structure also increases.  


{\bf Indentation:} Hindle \etal \cite{Hindle2008ICPC} created a complexity
measure that can be used to measure the complexity of the entire system or of a
change by examining the level of indentation on source lines. We calculate
their measure for reviews. They found that the strongest predictors of
complexity were the sum and standard deviation of the indentation of source
lines. Since our contribution sizes are so small, see Section
\ref{section:size}, the standard deviation is not meaningful, so we just sum
the amount of indentation.
%
%Furthermore, their measure is dependent on the norms in the project and the
%programming language.  as different levels of indentation may be used.  indent
The sum of indentation correlates strongly with churn (min (fb) $= .78$, mean
$= .89$, max (lx) $= .94$), so it is dropped from further analysis.

When one or more measures are highly correlated, parsimony requires us to drop
the more complex measures in our model. Churn, see Section \ref{section:size},
is kept over indentation and number of change blocks, and the number of
modified files in a change is kept over directory distance of files in a
change. 
 
In conclusion, it is intuitive that a large amount of churn and modified files
would affect a large number and diverse sections of the system and thus
represent a complex change. It is also reasonable that the more diffs generated
during a review the more complex the change.
%
In agreement with intuition and with previous literature comparing other
complexity measures with the number of lines in a file or system, we find that
the number of modified files and the churn correlate strongly with other
proposed measures of complexity and seem to provide a reasonable proxy for the
complexity of a review. Thus, we keep these two simple measures of complexity
and the number of diffs per review, and we discard the other four more complex,
\ie more difficult to calculate, measures.


